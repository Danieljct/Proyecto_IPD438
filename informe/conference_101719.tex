\documentclass[conference]{IEEEtran}
\IEEEoverridecommandlockouts
% El preceding line is only needed to identify funding in the first footnote. If that is unneeded, please comment it out.
\usepackage{cite}
\usepackage{amsmath,amssymb,amsfonts}
\usepackage{algorithmic}
\usepackage{graphicx}
\usepackage{textcomp}
\usepackage{xcolor}
\usepackage{hyperref}
\usepackage{subcaption} % Para subfiguras

\def\BibTeX{{\rm B\kern-.05em{\sc i\kern-.025em b}\kern-.08em
    T\kern-.1667em\lower.7ex\hbox{E}\kern-.125emX}}
\begin{document}

\title{Replicación del Sistema de Monitoreo de Red \\ Microsegundo-Nivel $\mu$MON: WaveSketch y Event Replay}

\author{\IEEEauthorblockN{Daniel Josué Cubillos Tapia}
\IEEEauthorblockA{\textit{Departamento de Electrónica} \\
\textit{Universidad Técnica Federico Santa María}\\
Valparaíso, Chile \\
daniel.cubillost@usm.cl}
}

\maketitle

\begin{abstract}
El monitoreo de red en centros de datos modernos requiere precisión a nivel de microsegundos ($\mu$s) para capturar micro-ráfagas y fluctuaciones de tasa de flujo. El sistema $\mu$MON propone una solución que utiliza el algoritmo \textit{WaveSketch} para la compresión eficiente de curvas de tasa de flujo (\textit{$\mu$Flow}) en los \textit{hosts}, y la detección de eventos de congestión (\textit{$\mu$Event}) mediante marcas ECN en los \textit{switches}. Este informe presenta la metodología, arquitectura e implementación de la replicación de $\mu$MON utilizando el simulador NS-3. Específicamente, replicamos la topología Fat-Tree $k=4$, el mecanismo de congestión ECN/RED, y el proceso de reconstrucción de tasas de flujo mediante wavelets. Los resultados demuestran la fidelidad del enfoque de monitoreo microsegundo-nivel y se valida la capacidad de \textit{Event Replay} al correlacionar la tasa de flujo reconstruida con las marcas de congestión.
\end{abstract}

\begin{IEEEkeywords}
Monitoreo de red, NS-3, $\mu$MON, WaveSketch, Fat-Tree, ECN, simulación.
\end{IEEEkeywords}

\section{Introducción}
\label{sec:introduction}
% (10 pts. Introducción)
Esta sección establece el \textbf{Problema y la Motivación} (10 pts. según la pauta).
\subsection{Problema y Motivación}
Describa la necesidad crítica de monitoreo a nivel de microsegundos en los centros de datos (micro-ráfagas, latencia, ajustes rápidos de CC).
Mencione que los sistemas tradicionales son insuficientes (milésimas de segundo).
\subsection{Objetivo del Informe}
Establezca claramente que el objetivo es replicar los componentes centrales del paper ``$\mu$MON: Empowering Microsecond-level Network Monitoring with Wavelets'' utilizando NS-3, enfocándose en la simulación de \textit{$\mu$Flow Measurement} y \textit{$\mu$Event Detection}.

\section{Trabajos Relacionados}
\label{sec:related_work}
% (5 pts. Trabajos relacionados)
Revisión concisa de soluciones de monitoreo de red relevantes, comparando su granularidad temporal y plataforma de implementación.
\begin{itemize}
    \item \textbf{Monitoreo de Granularidad Gruesa:} Mencione Netflow/SNMP (segundos/minutos).
    \item \textbf{Monitoreo a Milisegundos:} Mencione soluciones de \textit{sketching} que no manejan compresión temporal (ej., Count-Min Sketch tradicional).
    \item \textbf{Monitoreo $\mu$s en Plano de Datos:} Mencione enfoques que requieren \textit{switches} programables (P4) como BurstRadar o ConQuest, destacando que $\mu$MON busca compatibilidad con \textit{commodity switches} (ECN/Mirroring).
\end{itemize}

\section{Arquitectura y Metodología de Replicación}
% (15 pts. estrategia y arquitectura)
Aquí se detalla la \textbf{Estrategia y Arquitectura} de la replicación.
\subsection{Arquitectura del Sistema $\mu$MON}
Presente un diagrama conceptual (si es posible) de la arquitectura $\mu$MON: Hosts (WaveSketch) $\leftrightarrow$ Switches (ECN/Mirroring) $\leftrightarrow$ Analizador ($\mu$Event Replay). \subsection{Diseño de la Topología en NS-3}
\begin{itemize}
    \item \textbf{Topología Fat-Tree $k=4$:} Detalle la construcción (4 pods, 4 core switches, 16 hosts).
    \item \textbf{Configuración de Enlaces:} $\text{100 Gbps}$ y $1\mu s$ de delay por salto.
\end{itemize}

\subsection{Implementación de $\mu$Flow Measurement (WaveSketch)}
Describa cómo se adaptó la lógica de WaveSketch en el código NS-3 (\texttt{FlowRateLogger::WriteCsv}).
\begin{itemize}
    \item \textbf{Recolección de $\mu$Flows:} Explicar la ventana de agregación (ej. $8.192 \mu s$ o $1 \mu s$).
    \item \textbf{Compresión/Reconstrucción:} Describir el uso de la lógica \texttt{waveletScheme.count()} y \texttt{waveletScheme.rebuild()} para obtener la tasa reconstruida ($\hat{f}(t)$).
\end{itemize}

\subsection{Implementación de $\mu$Event Detection}
\begin{itemize}
    \item \textbf{Mecanismo ECN/RED:} Detallar la configuración de la cola (\texttt{RedQueueDisc}, \texttt{MinTh}, \texttt{MaxTh}, \texttt{UseEcn=true}). Mencionar la trazabilidad de la marca ECN (\texttt{TraceConnectWithoutContext("Mark", ...)}).
    \item \textbf{Parámetros de Congestión:} Especificar los umbrales utilizados (ej., $20\text{ KiB}$ y $200\text{ KiB}$ para KMin y KMax, si se ajustaron para reflejar el paper).
\end{itemize}

\section{Resultados y Análisis de Replicación}
% (10 pts. resultados)
Esta es la sección de \textbf{Resultados}.

\subsection{Fidelidad del Monitoreo $\mu$Flow}
\begin{figure}[htbp]
    \centering
    \begin{subfigure}[t]{0.45\textwidth}
        \centering
        \includegraphics[draft, width=\textwidth]{placeholder1.png} % Reemplazar con el gráfico real
        \caption{Tasa de flujo vs. Tasa reconstruida (WaveSketch).}
    \end{subfigure}
    \hfill
    \begin{subfigure}[t]{0.45\textwidth}
        \centering
        \includegraphics[draft, width=\textwidth]{placeholder2.png} % Reemplazar con el gráfico real
        \caption{Métricas de precisión (ARE, Cosine Similarity) vs. Memoria.}
    \end{subfigure}
    \caption{Resultados de la replicación de $\mu$Flow Measurement.}
    \label{fig:results_flow}
\end{figure}
\begin{itemize}
    \item \textbf{Comparación de Tasa:} Presente una gráfica (Figura \ref{fig:results_flow}a) de la curva de tasa de flujo original ($f(t)$, de NS-3) contra la tasa reconstruida ($\hat{f}(t)$, de WaveSketch). Analice visualmente la fidelidad.
    \item \textbf{Análisis de Precisión:} Presente una tabla o gráfica (Figura \ref{fig:results_flow}b) de las métricas de precisión (ARE, Cosine Similarity, etc.) obtenidas, comparándolas con los rangos reportados en el paper original (ej., Figuras 11 y 12 del paper).
\end{itemize}

\subsection{$\mu$Event Replay (Correlación de Congestión)}
\begin{figure}[htbp]
    \centering
    \includegraphics[draft, width=0.8\columnwidth]{placeholder3.png} % Reemplazar con el gráfico real
    \caption{Replicación del $\mu$Event Replay: Correlación de la Tasa de Flujo con las Marcas ECN.}
    \label{fig:event_replay}
\end{figure}
\begin{itemize}
\item Presente el resultado del \textit{Event Replay} (Figura \ref{fig:event_replay}).
\item Muestre un segmento de la curva de tasa de flujo de un flujo y destaque (sombreando) las ventanas de tiempo donde se detectaron \textbf{\texttt{ecn\_marks}} ($\mu \text{Events}$).
\item Analice cómo la tasa de flujo (el flujo víctima) reacciona inmediatamente después de la marca ECN, validando el concepto de \textit{Event Replay} (similar a la Figura 10c del paper).
\end{itemize}

\section{Conclusiones y Trabajo Futuro}
% (10 pts. Conclusiones y trabajo futuro)
\subsection{Conclusiones}
Resuma los hallazgos principales. Confirme si la replicación del enfoque WaveSketch y el Event Replay fueron exitosos (ej., "WaveSketch logra una alta similitud de energía (>$0.95$) con una compresión significativa").
\subsection{Trabajo Futuro}
Mencione posibles extensiones:
\begin{itemize}
    \item Implementación de la versión completa de WaveSketch (parte \textit{heavy/light}).
    \item Simulación del overhead de ancho de banda del monitoreo.
    \item Integración de un algoritmo de congestión como DCQCN para un escenario de RDMA más fiel al paper.
\end{itemize}

\section*{Agradecimientos}
(Opcional, si corresponde).

\section*{Referencias}
% (5 pts. Referencias, con buen uso de citas)
Liste todas las referencias. La pauta requiere un buen uso de citas en el texto.
\begin{thebibliography}{00}
\bibitem{b1} Hao Zheng, et al. ``$\mu$MON: Empowering Microsecond-level Network Monitoring with Wavelets.'' ACM SIGCOMM 2024. [El paper replicado]
\bibitem{b2} Una referencia de revista relacionada (ej. sobre ECN/DCTCP, como el paper de Alizadeh et al. o Zhu et al.).
\bibitem{b3} Una referencia sobre la topología o el simulador (NS-3).
\bibitem{b4} Otras referencias que haya citado en la sección de Trabajos Relacionados.
\end{thebibliography}

\end{document}